\documentclass[twocolumn]{jsarticle}
\usepackage{amssymb,amsmath,amsthm}
\usepackage{newtxtt}
\usepackage[utf8]{inputenc}
\newcommand{\kakko}[1][]{(#1)}
\newcommand{\bx}{\bold{x}}
\newcommand{\bb}{\bold{b}}
\newcommand{\bd}{\bold{d}}


\date{\today}
\author{山田龍}
\title{}
\begin{document}
\maketitle
\section{Gaussの消去法}
\section{Gauss-Seidel法}
\begin{equation}
    A = D + L + U
\end{equation}
の順番で対角成分、下三角成分、上三角成分を定義する。
解くべき連立方程式は、
\begin{align}
  (D + L + U)\bx = \bb  \\
  \bx = -D^{-1}(L + U)\bx +  D^{-1} \bb 
\end{align}
ヤコビ法では、以下の漸化式が成立すれば$x^{\kakko[k]}$は連立方程式の解である。
\begin{equation}
    \bx^{\kakko[k+1]} = - D^{-1} L \bx^{\kakko[k]}\\
    - D^{-1} U \bx^{\kakko[k]} + D^{-1}\bold{b} 
\end{equation}
Gauss-Seidel法では、以下の漸化式が成立すれば$x^{\kakko[k]}$は連立方程式の解である。
\begin{align}
    \bx^{\kakko[k+1]} &= - D^{-1} L \bx^{\kakko[k+1]}
    - D^{-1} U \bx^{\kakko[k]} + D^{-1}\bold{b}\\
    &=- D^{-1} (L + U) \bx^{\kakko[k]} + D^{-1}\bold{b} 
\end{align}
反復の計算は、
\begin{align*}
    (E + D^{-1}L)\bx^{\kakko[k+1]} = - D^{-1} U \bx^{\kakko[k]} + D^{-1}\bold{b}\\
    (D + L)D^{-1}\bx^{\kakko[k+1]} = - D^{-1} U \bx^{\kakko[k]} + D^{-1}\bold{b}\\
    \bx^{\kakko[k+1]} = - (D + L)^{-1}U \bx^{\kakko[k]} +(D + L)^{-1}\bold{b}\\
\end{align*}
ここで、$\bd^{\kakko[k+1]} = \bx^{\kakko[k+1]} - \bx^{\kakko[k]}$を定義する。
\begin{align*}
  \bd^{\kakko[k+1]} &= - \left[ (D + L)^{-1}U + E\right] \bx^{\kakko[k]} + (D + L)^{-1}\bb\\
  &= - \left[ (D + L)^{-1}U - D^{-1}U + D^{-1}U+ E\right] \bx^{\kakko[k]}\\
  &+ \left[(D + L)^{-1} - D^{-1} + D^{-1}\right]\bb\\
  &= - \left[ - L(D + L)^{-1}D^{-1}U + D^{-1}U+ E\right] \bx^{\kakko[k]}\\
  &+ \left[- L(D + L)^{-1}D^{-1}+ D^{-1}\right]\bb\\
  &= - D^{-1}L \bx^{\kakko[k+1]}- \left( D^{-1}U+ E\right) \bx^{\kakko[k]} + D^{-1}\bb
\end{align*}
更に変形する。
\begin{align*}
  (E + D^{-1}L)\bd^{\kakko[k+1]} \\ 
& = - D^{-1}L (- D^{-1} U \bx^{\kakko[k]} + D^{-1}\bold{b})\\
&- \left( D^{-1}U+ E\right) (- D^{-1} U \bx^{\kakko[k-1]} + D^{-1}\bold{b})\\
&+ (D^{-1} + D^{-1} L D^{-1}) \bb\\
&=  - D^{-1}L D^{-1} U \bx^{\kakko[k]} - D^{-1}L D^{-1}\bold{b}\\
&+ \left( D^{-1}UD^{-1} U+ D^{-1} U\right)  \bx^{\kakko[k-1]} \\
&-( D^{-1}UD^{-1} + D^{-1})\bold{b}\\
&+ (D^{-1} + D^{-1} L D^{-1})\bb
\end{align*}
両辺変形して、
\begin{align*}
D^{-1}(D + L)\bd^{\kakko[k+1]} &= D^{-1} [-L D^{-1} U \bx^{\kakko[k]} - L D^{-1}\bold{b}\\
&+ \left( UD^{-1} U+ U\right)  \bx^{\kakko[k-1]} \\
&+(-UD^{-1} + LD^{-1})\bold{b}]\\
(D + L)\bd^{\kakko[k+1]}&= - L D^{-1} U \bx^{\kakko[k]} - L D^{-1}\bold{b}\\
&+ \left( UD^{-1} U+ U\right)  \bx^{\kakko[k-1]} \\
& + (-UD^{-1} + LD^{-1})\bold{b}\\
\end{align*}
左辺の係数の逆行列を両辺にかけて、
\begin{align*}
\bd^{\kakko[k+1]} &= (D + L)^{-1}U[- L D^{-1}  \bx^{\kakko[k]} + \left( D^{-1} U+ E\right) \bx^{\kakko[k-1]}
-D^{-1}\bold{b}]\\
&= (D + L)^{-1}U\bd^{\kakko[k]}\\
\end{align*}
よってGauss-Seidel法において解が収束する条件は、
\begin{equation}
    ||(D + L)^{-1}U||_2 < 1
\end{equation}
\end{document}